\asciicmd{loglev\{i\}}{
    \index{user_cmds_index}{loglev}
    \label{cmd:loglev}

    \textbf{Name:} \texttt{loglev} -- Set the logger severity level.\\
    \textbf{Description:} System log messages more severe than the
    programmed severity level (\texttt{i}) will be logged.  All others
    will be muted.

    \begin{center}
      \begin{tabular}{|c|c|} \hline
        i   &Log level\\
        \hline \hline
        0   &ISR\\
        \hline
        1   &Information\\
        \hline
        2   &Warning\\
        \hline
        3   &Error\\

        \hline
      \end{tabular}
    \end{center}    

    \textbf{Example:}
    \boxtext{\texttt{loglev 2}
      \codecomment{Warnings and errors will be reported}\\
      \texttt{loglev 3}    \codecomment{Only error messages will be reported}
    }
    
    \nailnote{Error sounds are tied to the error levels.  If you set
      the log level to 3, you won't hear warning sounds.  You'll still
      hear error sounds.  Use the \texttt{logreg} command described
      \vpageref{cmd:logreg} to turn off all log messages and sounds.}


    \rednote{Setting the log level too low can make the system
      unstable.  For example, if the log level is set to ``ISR,'' the
      sytem will report every single character received via the remote
      interface.  The extra time spent reporting may cause the system
      to ignore incoming characters.}


}
