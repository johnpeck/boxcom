\section{Firmware}


\subsection{Received character flow}

Receiving remote commands begins with receiving characters. Reception
of these characters triggers an interrupt service routine (ISR), which
handles them according to the flow shown in figure
\ref{fig:recflow}. The first step in this flow is loading the
characters into the receive buffer.

\begin{figure}[ht]
  \begin{center}
    \includegraphics[clip,scale=0.8]{./tikz/recv_cmd_flow}
    \draftnote{recv\_cmd\_flow.tex}
    \caption{Program flow for processing characters received over the
      Butterfly's USART.  Sending a command terminator (carriage
      return) will always result in an empty receive buffer.  This is
      a good way to make sure there's no garbage in the buffer before
      writing to it.\label{fig:recflow}}
  \end{center}
\end{figure}

Figure \ref{fig:recbuffer} illustrates the receive buffer loaded with
a combined string.  The buffer is accessed with a pointer to its
beginning and a second pointing to the next index to be written.
These pointers are members of the \texttt{recv\_cmd\_state\_t} type
variable \texttt{recv\_cmd\_state}.  This is just style -- I like to
try to organize variables used in a flow by making them members of
their own structure.  Naming conventions aside, it's important to
notice that there are no limitations on command or argument size
imposed in this first step, provided that the total character count
stays under the \texttt{RECEIVE\_BUFFER\_SIZE} limit.
 
\begin{figure}[ht]
  \begin{center}
    \includegraphics[clip,scale=1.1]{recbuffer}
    \draftnote{recbuffer.fig}
    \caption{The received character buffer and pointers used to fill
      it.  There is no limit to the size of commands and their
      arguments, as long as the entire combined string and terminator
      fit inside \texttt{RECEIVE\_BUFFER\_SIZE}.\label{fig:recbuffer}}
  \end{center}
\end{figure}



When a combined string in the receive buffer is finished with a
carriage return, the string is copied over to a second buffer.  I call
this the parse buffer, since this is where the string will be searched
for recognized commands and arguments.  This buffer is locked until
its contents can be processed to keep it from being clobbered by new
combined strings.  Sending commands faster than they can be processed
will generate an error, and combined strings sent to a locked parse
buffer will be dropped.  The maximum command processing frequency will
depend on the system clock and other system tasks.  Not having larger
parse or receive buffers is a limitation that puts this project at the
hobby level.  Extending these buffers to hold more than just one
command would make the system more robust.

\clearpage{}
\subsection{How remote commands are processed}

After combined strings are copied from the receive to the parse
buffer, the system separates them into command and argument strings
using the flow shown in figure \ref{fig:cmdflow}.  Commands in the
parse buffer are then separated from their arguments with a string
terminator inserted into the first space between the two.  As
illustrated in figure \ref{fig:prsbuffer}, pointers to the beginning
of the parse buffer and the beginning of the argument will then
reference two separate strings.  The first of these two, the command
string, is converted to lowercase and compared with those in the
command definition array to look for a match.

% The parse buffer processing flowchart
\begin{figure}[ht]
  \begin{center}
    \includegraphics[clip,scale=0.7]{./tikz/parse_cmd_flow}
    \draftnote{parse\_cmd\_flow.tex}
    \caption{Program flow for processing fully-formed commands.  This
      processing step happens in the main loop, so it will only run
      when the system isn't busy doing something else.  Notice that
      all incoming commands are converted to lowercase, so there's no
      case sensitivity.\label{fig:cmdflow}}
  \end{center}
\end{figure}


% The parse buffer figure showing command and argument strings
\begin{figure}[ht]
  \begin{center}
    \includegraphics[clip,scale=1]{prsbuffer}
    \draftnote{prsbuffer.fig}
    \caption{The parse buffer with pointers to the command and
      argument strings.  The single combined string copied from the
      receive buffer is made into two strings by inserting a string
      terminator.\label{fig:prsbuffer}}
  \end{center}
\end{figure}






\clearpage{}
\subsection{Adding a new remote command}
\newcounter{comcount}

\setcounter{comcount}{1}
\paragraph{\arabic{comcount} --  Choose a name for the command}
The command characters, the argument characters, one space, and one
string terminator must all fit in the received command buffer. The
definition of this size is shown below.

\filesnip{bx\_command.h}
{ \texttt{\#define RECEIVE\_BUFFER\_SIZE 20} }

\addtocounter{comcount}{1}
\paragraph{\arabic{comcount} -- Think about the command's arguments}
The code can only handle unsigned hexadecimal number arguments
formatted as strings.  If you need something else, you'll have to
write more code.

\addtocounter{comcount}{1}
\paragraph{\arabic{comcount} -- Add a function for your command to call}
I like to put the new function in the module where it belongs, and
just add a ``cmd\_'' prefix to it.  If the command is a query, I add a
``\_q'' suffix.  The command corresponding to the \texttt{vcounts?}
query is shown below.

\begin{center}
  \vspace{-\baselineskip}
  \begin{tabular}{|l|} \hline
    \rowcolor[gray]{0.8}
    \begin{minipage}[c]{\textwidth - 2\tabcolsep}
      \textbf{File:}
      bx\_adc.c
    \end{minipage}\\
    \begin{minipage}[c]{\textwidth - 2\tabcolsep}
      \vspace{0.5\baselineskip}
      $\cdots$ \\
      \begin{minipage}[c]{\textwidth - 2\tabcolsep}
        \lstset{language=c}
        \begin{lstlisting}
void cmd_vcounts_q(uint16_t nonval) {
  uint16_t adc_temp = 0;
  adc_temp = adc_read();
  usart_printf_p(PSTR("0x%x\r\n"),adc_temp);
}
        \end{lstlisting}
      \end{minipage}
      $\cdots$\\
      \vspace{-0.5\baselineskip}
    \end{minipage}\\
    \hline
  \end{tabular}
\end{center}


\addtocounter{comcount}{1}
\paragraph{\arabic{comcount} -- Give the new command an entry in the command array}
A sample entry in the command array is shown below.  New entries must
be added before the ``end of table indicator.'' Remember that
hexadecimal arguments larger than 4 characters don't make sense for
16-bit integers (leading \texttt{0x} characters are not allowed).

\begin{center}
  \vspace{-\baselineskip}
  \begin{tabular}{|l|} \hline
    \rowcolor[gray]{0.8}
    \begin{minipage}[c]{\textwidth - 2\tabcolsep}
      \textbf{File:}
      bx\_command.c
    \end{minipage}\\
    \begin{minipage}[c]{\textwidth - 2\tabcolsep}
      \vspace{0.5\baselineskip}
      $\cdots$ \\
      \begin{minipage}[c]{\textwidth - 2\tabcolsep}
        \lstset{language=c}
        \begin{lstlisting}
command_t command_array[] ={
  // hello -- Print a greeting.
  {"hello",           // Name of the command
    "none",            // Argument type (can be "none" or "hex" right now)
    0,                 // Maximum number of characters in argument
    &cmd_hello,        // Address of function to execute
    helpstr_hello},    // The help text (defined above)  
  // End of table indicator.  Must be last.
  {"","",0,0,nullstr}
};      
\end{lstlisting}
      \end{minipage}
      $\cdots$\\
      \vspace{-0.5\baselineskip}
    \end{minipage}\\
    \hline
  \end{tabular}
\end{center}


\clearpage
\subsection{Logger functions}

% logger_msg_p
\subsubsection{\texttt{logger\_msg\_p}}
\onecodeline{void logger_msg_p( char *logsys,
  logger_level_t loglevel, const char *logmsg, ... );
}
\textsl{Send a message to the logger module from permanent memory}  


\paragraph{Parameters}
\begin{itemize}
\item \texttt{logsys}
  \subitem Pointer to a string matching one of the logger system strings.
\item \texttt{loglevel}
  \subitem One of the logger level identifiers:
  \subsubitem \verb8log_level_ISR8 (lowest level)
  \subsubitem \verb8log_level_INFO8
  \subsubitem \verb8log_level_WARNING8
  \subsubitem \verb8log_level_ERROR8 (highest level)
\item \texttt{logmsg} \subitem Pointer to a string stored in
  permanent (flash) memory.  This might be a C format string.
\item \texttt{...} \textsl{(additional arguments)} 
  \subitem Depending on the format string, the function may expect a
  sequence of additional arguments, each containing a value to be used
  to replace a format specifier in the format string.  There should be
  at least as many of these arguments as the number of values
  specified in the format specifiers. Additional arguments are ignored
  by the function.
\end{itemize}

\paragraph{Examples}

\begin{center}
  \begin{minipage}[c]{\textwidth}
    \lstset{language=c,backgroundcolor=\color{codegray},
      showstringspaces=false,breaklines=false}
    \begin{lstlisting}
 logger_msg_p("command",log_level_INFO,
	      PSTR("Command '%s' recognized.\r\n"),command_array -> name);
    \end{lstlisting}
  \end{minipage}
\end{center}
Output (After receiving the \texttt{hello} command):
\boxcmd{[I](command) Command 'hello' recognized.}

\clearpage{}
\subsection{Calibration factors}

\begin{table}[ht]
  \begin{center}
    \begin{tabular}{|c|c|}
      \hline
      Data type &Description\\
      \hline\hline
      \texttt{uint8\_t} &Unsigned 8-bit integer\\
      \hline
      \texttt{int8\_t}   &Signed 8-bit integer\\
      \hline
      \texttt{unit16\_t} &16-bit eeprom address\\
      \hline
      \end{tabular}
      \caption{The calibration factor structure will consist of two
        8-bit integers and a base address for the calibration factor
        in eeprom memory.  The eeprom memory locations will need to be
        two addresses apart.}
  \end{center}
\end{table}

