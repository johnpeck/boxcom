\section{Firmware}


\subsection{Adding a new remote command}
\newcounter{comcount}

\setcounter{comcount}{1}
\paragraph{\arabic{comcount} --  Choose a name for the command}
The command characters, the argument characters, one space, and one
string terminator must all fit in the received command buffer. The
definition of this size is shown below.

\filesnip{bx\_command.h}
{ \texttt{\#define RECEIVE\_BUFFER\_SIZE 20} }

\addtocounter{comcount}{1}
\paragraph{\arabic{comcount} -- Think about the command's arguments}
The code can only handle unsigned hexadecimal number arguments
formatted as strings.  If you need something else, you'll have to
write more code.

\addtocounter{comcount}{1}
\paragraph{\arabic{comcount} -- Add a function for your command to call}
I like to put the new function in the module where it belongs, and
just add a ``cmd\_'' prefix to it.  If the command is a query, I add a
``\_q'' suffix.  The command corresponding to the \texttt{vcounts?}
query is shown below.

\begin{center}
  \vspace{-\baselineskip}
  \begin{tabular}{|l|} \hline
    \rowcolor[gray]{0.8}
    \begin{minipage}[c]{\textwidth - 2\tabcolsep}
      \textbf{File:}
      bx\_adc.c
    \end{minipage}\\
    \begin{minipage}[c]{\textwidth - 2\tabcolsep}
      \vspace{0.5\baselineskip}
      $\cdots$ \\
      \begin{minipage}[c]{\textwidth - 2\tabcolsep}
        \lstset{language=c}
        \begin{lstlisting}
void cmd_vcounts_q(uint16_t nonval) {
  uint16_t adc_temp = 0;
  adc_temp = adc_read();
  usart_printf_p(PSTR("0x%x\r\n"),adc_temp);
}
        \end{lstlisting}
      \end{minipage}
      $\cdots$\\
      \vspace{-0.5\baselineskip}
    \end{minipage}\\
    \hline
  \end{tabular}
\end{center}


\addtocounter{comcount}{1}
\paragraph{\arabic{comcount} -- Give the new command an entry in the command array}
A sample entry in the command array is shown below.  New entries must
be added before the ``end of table indicator.'' Remember that
hexadecimal arguments larger than 4 characters don't make sense for
16-bit integers (leading \texttt{0x} characters are not allowed).

\begin{center}
  \vspace{-\baselineskip}
  \begin{tabular}{|l|} \hline
    \rowcolor[gray]{0.8}
    \begin{minipage}[c]{\textwidth - 2\tabcolsep}
      \textbf{File:}
      bx\_command.c
    \end{minipage}\\
    \begin{minipage}[c]{\textwidth - 2\tabcolsep}
      \vspace{0.5\baselineskip}
      $\cdots$ \\
      \begin{minipage}[c]{\textwidth - 2\tabcolsep}
        \lstset{language=c}
        \begin{lstlisting}
command_t command_array[] ={
  // hello -- Print a greeting.
  {"hello",           // Name of the command
    "none",            // Argument type (can be "none" or "hex" right now)
    0,                 // Maximum number of characters in argument
    &cmd_hello,        // Address of function to execute
    helpstr_hello},    // The help text (defined above)  
  // End of table indicator.  Must be last.
  {"","",0,0,nullstr}
};      
\end{lstlisting}
      \end{minipage}
      $\cdots$\\
      \vspace{-0.5\baselineskip}
    \end{minipage}\\
    \hline
  \end{tabular}
\end{center}


\clearpage
\subsection{Logger functions}

% logger_msg_p
\subsubsection{\texttt{logger\_msg\_p}}
\onecodeline{void logger_msg_p( char *logsys,
  logger_level_t loglevel, const char *logmsg, ... );
}
\textsl{Send a message to the logger module from permanent memory}  


\paragraph{Parameters}
\begin{itemize}
\item \texttt{logsys}
  \subitem Pointer to a string matching one of the logger system strings.
\item \texttt{loglevel}
  \subitem One of the logger level identifiers:
  \subsubitem \verb8log_level_ISR8 (lowest level)
  \subsubitem \verb8log_level_INFO8
  \subsubitem \verb8log_level_WARNING8
  \subsubitem \verb8log_level_ERROR8 (highest level)
\item \texttt{logmsg} \subitem Pointer to a string stored in
  permanent (flash) memory.  This might be a C format string.
\item \texttt{...} \textsl{(additional arguments)} 
  \subitem Depending on the format string, the function may expect a
  sequence of additional arguments, each containing a value to be used
  to replace a format specifier in the format string.  There should be
  at least as many of these arguments as the number of values
  specified in the format specifiers. Additional arguments are ignored
  by the function.
\end{itemize}

\paragraph{Examples}

\begin{center}
  \begin{minipage}[c]{\textwidth}
    \lstset{language=c,backgroundcolor=\color{codegray},
      showstringspaces=false,breaklines=false}
    \begin{lstlisting}
 logger_msg_p("command",log_level_INFO,
	      PSTR("Command '%s' recognized.\r\n"),command_array -> name);
    \end{lstlisting}
  \end{minipage}
\end{center}
Output (After receiving the \texttt{hello} command):
\boxcmd{[I](command) Command 'hello' recognized.}

\clearpage{}
\subsection{Calibration factors}

\begin{table}[ht]
  \begin{center}
    \begin{tabular}{|c|c|}
      \hline
      Data type &Description\\
      \hline\hline
      \texttt{uint8\_t} &Unsigned 8-bit integer\\
      \hline
      \texttt{int8\_t}   &Signed 8-bit integer\\
      \hline
      \texttt{unit16\_t} &16-bit eeprom address\\
      \hline
      \end{tabular}
      \caption{The calibration factor structure will consist of two
        8-bit integers and a base address for the calibration factor
        in eeprom memory.  The eeprom memory locations will need to be
        two addresses apart.}
  \end{center}
\end{table}

